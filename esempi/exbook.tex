\documentclass[12pt]{book}              % Book class in 11 points
\parindent0pt  \parskip10pt             % make block paragraphs
\raggedright                            % do not right justify

\title{\bf An Example of Book Class}    % Supply information
\author{for \LaTeX\ Class}              %   for the title page.
\date{\today}                           %   Use current date. 

% Note that book class by default is formatted to be printed back-to-back.
\begin{document}                        % End of preamble, start of text.
\frontmatter                            % only in book class (roman page #s)
\maketitle                              % Print title page.
\tableofcontents                        % Print table of contents
\mainmatter                             % only in book class (arabic page #s)
%\part{A Part Heading}                   % Print a "part" heading
\chapter{A Main Heading}                % Print a "chapter" heading
Most of this example applies to \texttt{article} and \texttt{book} classes
as well as to \texttt{report} class. In \texttt{article} class, however,
the default position for the title information is at the top of
the first text page rather than on a separate page. Also, it is
not usual to request a table of contents with \texttt{article} class.
 
\section{A Subheading}                  % Print a "section" heading
The following sectioning commands are available:
\begin{quote}                           % The following text will be
 part \\                                %    set off and indented.
 chapter \\                             % \\ forces a new line
 section \\ 
 subsection \\ 
 subsubsection \\ 
 paragraph \\ 
 subparagraph 
\end{quote}                             % End of indented text
But note that---unlike the \texttt{book} and \texttt{report} classes---the
\texttt{article} class does not have a ``chapter" command.
 
\end{document}                          % The required last line
