%!TEX root = ../../tesi-ayurveda-v1.tex


\chapter{G\d{r}idhra\={s}i: sciatica}                  % Print a "section" heading

\section{Etimologia}


Gridhra + so – atonupasargitcha – Adding ‘kah’ pratya leads to
Gridhra + so + ka by lopa of ‘o’ and ‘k’, ‘s’ is replaced by ‘sa’ by rule
‘Dhatvadeh’ ‘sah sah’ ‘G\d{r}idhra\={s}i’ derived.
Gridhasi word is derived from ‘gridhna’ dhatu, meaning to
desire, to strive after greedily or to be eager for. By the rule of
‘Susudhangridhangridhi bhyaha kran’ (Unadi 2/24) by adding
‘karana’ pratyaya i.e. ‘gridh + kran’ by lope ‘k’ and ‘n’ the word ‘gridh +
ra’ the word ‘gridhra’ is derived.
Conceptual Contrive
12
Gridharam and Syati so Antakarmani Atonupasargakah,
Chanhva Gridhra Iva Syati Pidayati, Gridhra Iva Syati Bhaksati.
Gridra is bird called as vulture in English. This bird is fond of
meat and he eats flesh of an animal in such a fashion that he deeply
pierce his beak in the flesh then draws it out forcefully, exactly such
type of pain occurs in G\d{r}idhra\={s}i and hence the name.
Another meaning is a man who is striving after meat greedily
like Gridhra (vulture) is prone to get it and hence the name G\d{r}idhra\={s}i.
Further as in this disease the patient walks like the bird
Gridhra and his legs become tense and slightly curved so due to the
resemblance with the gait of a vulture, G\d{r}idhra\={s}i term might have
been given to this disease.


\section{Introduzione}

In Ayurveda, G\d{r}idhra\={s}i viene descritta come uno degli 80 tipi di Nanatmaja V\={a}tavy\={a}dhi.

In relazione al tipo V\={a}taja di G\d{r}idhra\={s}i, l'\={A}ch\={a}rya Charaka menziona - nel capitolo 28 del Chikitsa Sthana - Ruka (dolore), Toda (dolore acuto, penetrante), Stambha (rigidità) e Mruhuspandana (contrazioni) come sintomi presenti nelle zone posteriori (p\d{r}\d{s}\d{t}abh\={a}ga) della vita e dell'anca (ka\d{t}\={\i}), della schiena (p\d{r}\d{s}\d{t}ha) della coscia (\={u}ru), del ginocchio (j\={a}nu), del polpaccio (ja\.{n}gh\={a}) e del piede (p\={a}da). I sintomi citati partono dalla zona dell'anca (sphic) e si irradiano verso la parte inferiore del corpo.

Per il tipo V\={a}takaphaja di G\d{r}idhra\={s}i, i sintomi aggiuntivi indicati da Charaka sono: Tandr\={a} (fatica, pigrizia, fiacchezza), Gaurava (pesantezza) e Arocaka (perdita di appetito).

L'\={A}ch\={a}rya Sushruta indica il sintomo principale della patologia. Egli dice che quando i legamenti (ka\d{n}\d{d}ar\={a}) del tallone e di tutte le dita dei piedi sono afflitti da un V\={a}ta
 viziato, i movimenti degli arti inferiori divengono limitati. Questo viene riconosciuto come un segno importante per la diagnosi della patologia G\d{r}idhra\={s}i\cite{tesi1}. 

Secondo il saggio Harita, i pa\~{n}ca vayava (subdosa) principalmente viziati in G\d{r}idhra\={s}i sono Vy\={a}na e Ap\={a}na V\={a}ta. Il movimento (Gati), l'estensione (Prasarana), la flessione (Akunchana), l'alzare (Utkshepana) sono legati a Vy\={a}na Vaju. L'impedimento di queste azioni (Sakthikshepa Karma) indica la presenza di Vy\={a}na V\={a}ta Du\d{s}\d{t}i. 



On the basis of sign and symptoms of Gridhrasi given in
Ayurvedic text, it can be correlated with modern disease sciatica,
because in sciatica pain is found along the course of sciatica nerve
that is to say in the buttock, back of the thigh, out side and back of
the leg and outer border of the foot.

Here one thing is noticeable that,
symptomatology of sciatica is same as given in Charaka Samhita. In
sciatica, pricking pain is specific symptom which is aggravated by
coughing, sneezing and by sleeping in night due to stretching of
muscles and nerve. Patient is unable to keep the leg straight that is
Sakthikshepa Nigraha.


Gridhrasi is a disorder, results from vitiation of Vata and this
Vata of Ayurveda can be correlated with nervous system of modern
science. Because in Ayurveda, it has been said that Vata is
responsible for the act of body viz. Praspandana, Udvahana, Purana,
Viveka, Dhrana (Su. Su. 15/1) and same on other hand according to
modern science, nervous system is responsible for all these body acts.
On account of aforementioned description, it is clear that
Gridhrasi is result of vitiation of Vyana Vayu and can be broadly
correlate with sciatica in latest medical science.


\section{Nidana Panchaka}


\subsection{Nidana}

\subsection{Purvarupa}

\subsection{Rupa}

\subsection{Upashaya-Anupashaya}

     Upashaya Ahara: Godhuma, Masha, Puranashali, Patol, Vartak, Kilata, Rasona, Taila, Ghrita, Kshira, Tila, Draksha, Dadima 

     Upashaya Vihara: Abhyanga, Tarpana, Swedana, NirV\={a}ta Sthana, Atapa, Sevana, Nasya, Ushnapravarana, Basti

     Anupashaya Ahara: Mudga, Kalaya, Brihatshali, Yava, Rajmasha, Kodrava, Kshara, sapore aspro e astringente 

     Anupashaya Vihara: Chinta, Bhaya, Shoka, Krodha, Vegavidharana, Chankramana, Annasana, Ativyavaya, Jagarana


\subsection{Samprapti}


\subsubsection{Do\d{s}a
}

   
   Dushya
   Srotasa
   Srotodu\d{s}\d{t}i
 Prakara
   Agni
   Ama
   Udbhavasthana
   Sanchara Sthana
   Adhisthana
   Vyakta Rupa


\section{Sapeksha Nidana: comparazione con disturbi simili}
 
\section{Sadhyata-Asadhyata: prognosi}

\section{Upadravas}


