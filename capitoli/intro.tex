%!TEX root = ../tesi-ayurveda-v1.tex

%----------------------------------------------------------------------------------------
%	HEADING SECTIONS
%----------------------------------------------------------------------------------------



\chapter*{Introduzione} 



Lista simboli \newline
\'{s}:  \textbackslash'\{s\} \newline
\={a}: \textbackslash=\{a\} \newline
\={\i}: \textbackslash=\{\textbackslash i\}  \newline
\d{a}:  \textbackslash d\{a\} \newline
\.{n}:   \textbackslash.\{n\}  \newline
\~{n}:    \textbackslash\textasciitilde\{n\} \newline
\d{\={R}}: \textbackslash d \{\textbackslash=\{R\}\} \newline

Most of this example applies to \texttt{article} and \texttt{book} classes
as well as to \texttt{report} class. In \texttt{article} class, however,
the default position for the title information is at the top of
the first text page rather than on a separate page. Also, it is
not usual to request a table of contents with \texttt{article} class.

Lo scritto si compone di due parti, suddivise a loro volta in 7 capitoli.
La prima parte illustra 

La seconda parte presenta la revisione delle ricerche scientifiche per rispondere al mio quesito seguendo la metodologia scientifica.
Ho rilevato i risultati e li ho analizzati il più oggettivamente possibile osservando l’applicabilità nella pratica della CBT al paziente affetto da CBLP.