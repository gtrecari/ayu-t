% -*- coding: utf-8 -*-
\documentclass[a4paper, openright, twoside, 12pt]{book}    


\usepackage{epigraph}

\usepackage[utf8]{inputenc}

\usepackage{mathptmx}
\usepackage[T1]{fontenc}
\usepackage[italian]{babel}   
\usepackage[printonlyused]{acronym}


\usepackage[a4paper,top=2cm,bottom=2cm,left=2cm,right=2cm]{geometry} 

\usepackage{geometry}
\usepackage{fancyhdr}
%\usepackage[nottoc]{tocbibind} per includere lista figure in indice
% per avere l'environment "comment" ------------------------------------
%\usepackage{verbatim}
%\usepackage{graphicx}
%\usepackage{makeidx}
%\usepackage{mathptmx}


\renewcommand{\epigraphsize}{\small}

%\setlength{\epigraphwidth}{0.8\textwidth}

\renewcommand{\textflush}{flushright} \renewcommand{\sourceflush}{flushright}

\let\originalepigraph\epigraph 
\renewcommand\epigraph[2]{\originalepigraph{\textit{#1}}{\textsc{#2}}}


\fancyfoot[RE,RO]{\thepage} % Left side on Even pages; Right side on Odd pages
\cfoot{}
\pagestyle{fancy}
\fancypagestyle{plain}{%
  \fancyhf{}%
  \renewcommand{\headrulewidth}{0pt}%
  \fancyhf[lef,rof]{\thepage}%
}


\linespread{1.3}

\begin{document}

\frontmatter


%----------------------------------------------------------------------------------------
%	PRIMA PAGINA
%----------------------------------------------------------------------------------------

\input{./template/copertina.tex}

%----------------------------------------------------------------------------------------
%	DEDICA
%----------------------------------------------------------------------------------------


%!TEX root = ../tesi-ayurveda-v1.tex

\begin{verse}
\begin{flushright}

\vspace{10cm}

\emph{"A me stessa, che mi spingo con entusiasmo e curiosità nei sentieri misteriosi della vita"}

\bigskip

\emph{"Alla Danza e all'\={A}yurveda, che mi accompagnano in questo cammino e come due pilastri mi hanno sostenuta nei momenti più difficili"}


\bigskip

\emph{"Ai miei cari senza i quali perderei il senso"}

\bigskip

\emph{"Ai Maestri che hanno abbracciato i miei dolori"}

\bigskip


\emph{"Alla vita"}


\end{flushright}
\end{verse}



%----------------------------------------------------------------------------------------
%	INDICE
%----------------------------------------------------------------------------------------


\tableofcontents

%\addcontentsline{toc}{chapter}{\listfigurename}
%\addcontentsline{toc}{chapter}{\listtablename}

\addcontentsline{toc}{chapter}{Abbreviazioni}

\addcontentsline{toc}{chapter}{Introduzione}

%\listoffigures
%\listoftables
 
\mainmatter


%----------------------------------------------------------------------------------------
%	ABBREVIAZIONI
%----------------------------------------------------------------------------------------

 %!TEX root = ../tesi-ayurveda-v1.tex


\chapter*{Abbreviazioni} 


\begin{acronym}
\acro{A.H.}{Ashtanga Hridaya}
\acro{A.S.}{Ashtanga Samgraha}
\acro{Ch.}{Charaka Samhita}
\acro{Chi.}{Chikitsa Sthana}
\acro{Su.}{Sushruta Samhita}
\end{acronym}




%----------------------------------------------------------------------------------------
%	INTRODUZIONE
%----------------------------------------------------------------------------------------

 
 %!TEX root = ../tesi-ayurveda-v1.tex

%----------------------------------------------------------------------------------------
%	HEADING SECTIONS
%----------------------------------------------------------------------------------------



\chapter*{Introduzione} 



Lista simboli \newline
\'{s}:  \textbackslash'\{s\} \newline
\={a}: \textbackslash=\{a\} \newline
\={\i}: \textbackslash=\{\textbackslash i\}  \newline
\d{a}:  \textbackslash d\{a\} \newline
\.{n}:   \textbackslash.\{n\}  \newline
\~{n}:    \textbackslash\textasciitilde\{n\} \newline
\d{\={R}}: \textbackslash d \{\textbackslash=\{R\}\} \newline

Most of this example applies to \texttt{article} and \texttt{book} classes
as well as to \texttt{report} class. In \texttt{article} class, however,
the default position for the title information is at the top of
the first text page rather than on a separate page. Also, it is
not usual to request a table of contents with \texttt{article} class.

Lo scritto si compone di due parti, suddivise a loro volta in 7 capitoli.
La prima parte illustra 

La seconda parte presenta la revisione delle ricerche scientifiche per rispondere al mio quesito seguendo la metodologia scientifica.
Ho rilevato i risultati e li ho analizzati il più oggettivamente possibile osservando l’applicabilità nella pratica della CBT al paziente affetto da CBLP.

%----------------------------------------------------------------------------------------
%****************************************************************************************
%----------------------------------------------------------------------------------------

%	PARTE I : La visione Ayurvedica delle patologie del rachide lombare

%----------------------------------------------------------------------------------------
%****************************************************************************************
%----------------------------------------------------------------------------------------


\part{La visione Ayurvedica delle patologie del rachide lombare}



%----------------------------------------------------------------------------------------
% PARTE I - CAPITOLO 1 - PATOLOGIE RACHIDE LOMBARE SECONDO LA MEDICINA MODERNA
%----------------------------------------------------------------------------------------


\input{./capitoli/part1/pat-rachide-lombare-moderna.tex}




%----------------------------------------------------------------------------------------
% PARTE I - CAPITOLO 2 - VATA VYADHI
%----------------------------------------------------------------------------------------


%!TEX root = ../../tesi-ayurveda-v1.tex

%----------------------------------------------------------------------------------------
%	HEADING SECTIONS
%----------------------------------------------------------------------------------------

\chapter{V\={a}ta Vy\={a}dhi}


\setlength{\epigraphwidth}{0.5\textwidth}

\epigraph{v\={a}yur\={a}yu rbala\d{m} v\={a}yu rv\={a}yu rdh\={a}t\={a} \'{s}ar\={\i}ri\d{n}\={a}m|
v\={a}yurvi\'{s}vamida\d{m} sarva\d{m} prabhurv\={a}yu\'{s}ca k\={\i}rtita\d{h} \\ (V\={a}ta
 do\d{s}a è la vita, è la forza, è il sostenimento del corpo, regge il corpo e la vita insieme. V\={a}ta
 è pervasivo, è il controllore di ogni cosa nell'universo)}{Caraka Sa\d{m}hit\={a}, Chi. 28/3}

 
 \section{Introduzione}

 \section{Eziologia}
  
  	\subsection{Aharataha}


  	\subsection{Viharataha}
  	

  	\subsection{Agantuka}


  	\subsection{Anya Hetu}


 \section{V\={a}taprakopa}


\ac{Su.}

\ac{A.H.}
\ac{A.S.}
\ac{Ch.}
\ac{Chi.}
\ac{Su.}




%----------------------------------------------------------------------------------------
% PARTE I - CAPITOLO 3 - KATISHOOLA
%----------------------------------------------------------------------------------------


%!TEX root = ../../tesi-ayurveda-v1.tex

%----------------------------------------------------------------------------------------
%	HEADING SECTIONS
%----------------------------------------------------------------------------------------


\chapter{Ka\d{t}\={\i} \'{S}\={u}la: lombalgia}                  % Print a "section" heading


\setlength{\epigraphwidth}{0.6\textwidth}

\epigraph{La schiena è la parte che non puoi vederti, quella che lasci agli altri. Sulla schiena pesano i pensieri, le spalle che hai voltato quando hai deciso di andartene}{Margaret Mazzantini, Venuto al mondo, 2012}




%----------------------------------------------------------------------------------------
% PARTE I - CAPITOLO 4 - GRIDHRASI
%----------------------------------------------------------------------------------------


%!TEX root = ../../tesi-ayurveda-v1.tex


\chapter{G\d{r}idhra\={s}i: sciatica}                  % Print a "section" heading

\section{Etimologia}


Gridhra + so – atonupasargitcha – Adding ‘kah’ pratya leads to
Gridhra + so + ka by lopa of ‘o’ and ‘k’, ‘s’ is replaced by ‘sa’ by rule
‘Dhatvadeh’ ‘sah sah’ ‘G\d{r}idhra\={s}i’ derived.
Gridhasi word is derived from ‘gridhna’ dhatu, meaning to
desire, to strive after greedily or to be eager for. By the rule of
‘Susudhangridhangridhi bhyaha kran’ (Unadi 2/24) by adding
‘karana’ pratyaya i.e. ‘gridh + kran’ by lope ‘k’ and ‘n’ the word ‘gridh +
ra’ the word ‘gridhra’ is derived.
Conceptual Contrive
12
Gridharam and Syati so Antakarmani Atonupasargakah,
Chanhva Gridhra Iva Syati Pidayati, Gridhra Iva Syati Bhaksati.
Gridra is bird called as vulture in English. This bird is fond of
meat and he eats flesh of an animal in such a fashion that he deeply
pierce his beak in the flesh then draws it out forcefully, exactly such
type of pain occurs in G\d{r}idhra\={s}i and hence the name.
Another meaning is a man who is striving after meat greedily
like Gridhra (vulture) is prone to get it and hence the name G\d{r}idhra\={s}i.
Further as in this disease the patient walks like the bird
Gridhra and his legs become tense and slightly curved so due to the
resemblance with the gait of a vulture, G\d{r}idhra\={s}i term might have
been given to this disease.


\section{Introduzione}

In Ayurveda, G\d{r}idhra\={s}i viene descritta come uno degli 80 tipi di Nanatmaja V\={a}tavy\={a}dhi.

In relazione al tipo V\={a}taja di G\d{r}idhra\={s}i, l'\={A}ch\={a}rya Charaka menziona - nel capitolo 28 del Chikitsa Sthana - Ruka (dolore), Toda (dolore acuto, penetrante), Stambha (rigidità) e Mruhuspandana (contrazioni) come sintomi presenti nelle zone posteriori (p\d{r}\d{s}\d{t}abh\={a}ga) della vita e dell'anca (ka\d{t}\={\i}), della schiena (p\d{r}\d{s}\d{t}ha) della coscia (\={u}ru), del ginocchio (j\={a}nu), del polpaccio (ja\.{n}gh\={a}) e del piede (p\={a}da). I sintomi citati partono dalla zona dell'anca (sphic) e si irradiano verso la parte inferiore del corpo.

Per il tipo V\={a}takaphaja di G\d{r}idhra\={s}i, i sintomi aggiuntivi indicati da Charaka sono: Tandr\={a} (fatica, pigrizia, fiacchezza), Gaurava (pesantezza) e Arocaka (perdita di appetito).

L'\={A}ch\={a}rya Sushruta indica il sintomo principale della patologia. Egli dice che quando i legamenti (ka\d{n}\d{d}ar\={a}) del tallone e di tutte le dita dei piedi sono afflitti da un V\={a}ta
 viziato, i movimenti degli arti inferiori divengono limitati. Questo viene riconosciuto come un segno importante per la diagnosi della patologia G\d{r}idhra\={s}i\cite{tesi1}. 

Secondo il saggio Harita, i pa\~{n}ca vayava (subdosa) principalmente viziati in G\d{r}idhra\={s}i sono Vy\={a}na e Ap\={a}na V\={a}ta. Il movimento (Gati), l'estensione (Prasarana), la flessione (Akunchana), l'alzare (Utkshepana) sono legati a Vy\={a}na Vaju. L'impedimento di queste azioni (Sakthikshepa Karma) indica la presenza di Vy\={a}na V\={a}ta Du\d{s}\d{t}i. 



On the basis of sign and symptoms of Gridhrasi given in
Ayurvedic text, it can be correlated with modern disease sciatica,
because in sciatica pain is found along the course of sciatica nerve
that is to say in the buttock, back of the thigh, out side and back of
the leg and outer border of the foot.

Here one thing is noticeable that,
symptomatology of sciatica is same as given in Charaka Samhita. In
sciatica, pricking pain is specific symptom which is aggravated by
coughing, sneezing and by sleeping in night due to stretching of
muscles and nerve. Patient is unable to keep the leg straight that is
Sakthikshepa Nigraha.


Gridhrasi is a disorder, results from vitiation of Vata and this
Vata of Ayurveda can be correlated with nervous system of modern
science. Because in Ayurveda, it has been said that Vata is
responsible for the act of body viz. Praspandana, Udvahana, Purana,
Viveka, Dhrana (Su. Su. 15/1) and same on other hand according to
modern science, nervous system is responsible for all these body acts.
On account of aforementioned description, it is clear that
Gridhrasi is result of vitiation of Vyana Vayu and can be broadly
correlate with sciatica in latest medical science.


\section{Nidana Panchaka}


\subsection{Nidana}

\subsection{Purvarupa}

\subsection{Rupa}

\subsection{Upashaya-Anupashaya}

     Upashaya Ahara: Godhuma, Masha, Puranashali, Patol, Vartak, Kilata, Rasona, Taila, Ghrita, Kshira, Tila, Draksha, Dadima 

     Upashaya Vihara: Abhyanga, Tarpana, Swedana, NirV\={a}ta Sthana, Atapa, Sevana, Nasya, Ushnapravarana, Basti

     Anupashaya Ahara: Mudga, Kalaya, Brihatshali, Yava, Rajmasha, Kodrava, Kshara, sapore aspro e astringente 

     Anupashaya Vihara: Chinta, Bhaya, Shoka, Krodha, Vegavidharana, Chankramana, Annasana, Ativyavaya, Jagarana


\subsection{Samprapti}


\subsubsection{Do\d{s}a
}

   
   Dushya
   Srotasa
   Srotodu\d{s}\d{t}i
 Prakara
   Agni
   Ama
   Udbhavasthana
   Sanchara Sthana
   Adhisthana
   Vyakta Rupa


\section{Sapeksha Nidana: comparazione con disturbi simili}
 
\section{Sadhyata-Asadhyata: prognosi}

\section{Upadravas}






%----------------------------------------------------------------------------------------
% PARTE I - CAPITOLO 5 - SISTEMA ENERGETICO SOTTILE
%----------------------------------------------------------------------------------------


\input{./capitoli/part1/sistema-energetico-sottile.tex}

%----------------------------------------------------------------------------------------
%****************************************************************************************
%----------------------------------------------------------------------------------------

% PARTE 2 : Kati Cikitsa: terapie ayurvediche per le patologie del rachide lombare

%----------------------------------------------------------------------------------------
%****************************************************************************************
%----------------------------------------------------------------------------------------


\part{Ka\d{t}\={\i} Cikits\={a}: terapie ayurvediche per le patologie del rachide lombare}


%----------------------------------------------------------------------------------------
% PARTE II - CAPITOLO 6 - TERAPIE SNEHANA
%----------------------------------------------------------------------------------------


%!TEX root = ../../tesi-ayurveda-v1.tex


\chapter{Le terapie dell'amore (Snehana)}

\section{Tandabhyanga}

 \subsection{do\d{s}a
 e guna coinvolti}
 \subsection{indicazioni}
 \subsection{controindicazioni}
  \subsection{materiali di utilizzo}
  \subsection{sinergie con altri trattamenti}

\section{Le terapie Dh\={a}ra}

\section{Lepa}

\section{Pichu}

\section{Basti esterni}



%----------------------------------------------------------------------------------------
% PARTE II - CAPITOLO 7 - TERAPIE SVEDANA
%----------------------------------------------------------------------------------------


%!TEX root = ../../tesi-ayurveda-v1.tex



\chapter{Le terapie ayurvediche del calore (Svedana)}

\section{Podikizhi}

\section{Terapie ayurvediche con il vapore}

The following sectioning commands are available:
\begin{quote}                           % The following text will be
 part \\                                %    set off and indented.
 chapter \\                             % \\ forces a new line
 section \\ 
 subsection \\ 
 subsubsection \\ 
 paragraph \\ 
 subparagraph 
\end{quote}                             % End of indented text
But note that---unlike the \texttt{book} and \texttt{report} classes---the
\texttt{article} class does not have a ``chapter" command.
 




%----------------------------------------------------------------------------------------
% PARTE II - CAPITOLO 8 - STUDI CLINICI
%----------------------------------------------------------------------------------------


%!TEX root = ../../tesi-ayurveda-v1.tex

\chapter{Studi clinici}





%----------------------------------------------------------------------------------------
% PARTE II - CAPITOLO 9 - TRATTAMENTI ENERGETICI
%----------------------------------------------------------------------------------------


%!TEX root = ../../tesi-ayurveda-v1.tex

\chapter{Trattamenti energetici}



%----------------------------------------------------------------------------------------
% PARTE II - CAPITOLO 10 - TAILA E KASHAYA SHOOLAHARA AYURVEDICHE
%----------------------------------------------------------------------------------------


%!TEX root = ../../tesi-ayurveda-v1.tex


\chapter{G\d{r}idhra\={s}i
: sciatica}                  % Print a "section" heading



%----------------------------------------------------------------------------------------
% PARTE II - CAPITOLO 11 - RICETTE AYURVEDICHE
%----------------------------------------------------------------------------------------


%!TEX root = ../../tesi-ayurveda-v1.tex

\chapter{Ricette ayurvediche per Ka\d{t}\={\i} \'{S}\={u}la e G\d{r}idhra\={s}i}

Questo capitolo descrive alcune ricette

\section{Spezie e proprietà curative} 


\subsection{Assafetida} 


\subsection{Cannella} 


\subsection{Chiodi di Garofano} 


\subsection{Cumino} 

\subsection{Fieno Greco} 

\subsection{Finocchio} 

\subsection{Cumino} 

\subsection{Pepe Lungo} 

\subsection{Pepe Nero}

\subsection{Senape Indiana}


\section{Preparati ayurvedici} 

\subsection{La\'{s}una K\d{s}\={\i}rap\={a}ka} 

Lashuna ksheerapaka is one of the classical preparation mentioned in Ayurveda which helps to decrease the smell and pungent taste of garlic,retaining the medicinal qualities unaltered.



Most \ac{ny}, of this G\d{r}idhra\={s}i example applies to \texttt{article} and \texttt{book} classes
as well as to \texttt{report} class. In \texttt{article} class, however,
the default position for the title information is at the top of
the first text page rather than on a separate page. Also, it is
not usual to request \cite{uno} a table of contents with \texttt{article} class.

Critical Analysis of Gridhrasi Chikitsa
 

%----------------------------------------------------------------------------------------
% PARTE II - CAPITOLO 12 - CONCLUSIONI
%----------------------------------------------------------------------------------------


%!TEX root = ../../tesi-ayurveda-v1.tex

\chapter{Conclusioni}

Ciao



%----------------------------------------------------------------------------------------
%	BIBLIOGRAFIA
%----------------------------------------------------------------------------------------


%!TEX root = ../tesi-ayurveda-v1.tex

\begin{thebibliography}{3} 
\addcontentsline{toc}{chapter}{Bibliografia} 


%----------------------------------------------------------------------------------------
%	INTRO
%----------------------------------------------------------------------------------------


\bibitem{webi} Smith, J. (2005). \textit{Cherry Pie for Beginners}. Ayurvedic Point. Estrapolato da http://www.piesforeveryone.com

%----------------------------------------------------------------------------------------
%	GENERALE
%----------------------------------------------------------------------------------------


\bibitem{test} C. Tosto, A. Morandi et al. \textit{Dispense del Corso per Terapisti Ayurvedici}. Ayurvedic Point.

\bibitem{ch-chi} Charaka. \textit{Charaka Sahmita, Chikitsa Sthana}.


\bibitem{ninivaggi} F.J. Ninivaggi \textit{\={A}yurveda. Una medicina con una tradizione antica di seimila anni}. Ubaldini Editore - Roma. 2002.

\bibitem{iannaccone-ayurveda} E. Iannaccone \textit{Ayurveda. La medicina dell'armonia tra l'uomo e l'universo}. Tecniche Nuove Edizioni. 2006.

\bibitem{iannaccone-mente} E. Iannaccone \textit{I Disordini della Mente nell'\={A}yurveda}. Laksmi Edizioni. 2015.


%----------------------------------------------------------------------------------------
%	ALIMENTAZIONE
%----------------------------------------------------------------------------------------

\bibitem{bergnach} B. Bergnach, V. Ravasi. \textit{La saggezza nel cibo, l'\={A}yurveda e la cucina delle nostre nonne}. Curcu \& Genovese. 2016. 

\bibitem{joythi-nutrizione} Swami Joythimayananda. \textit{Alimentazione ayurvedica. Manuale per una nutrizione equilibrata e sana}. Macro Edizioni. 2016. 



%----------------------------------------------------------------------------------------
%	CANALI ENERGETICI SOTTILI 
%----------------------------------------------------------------------------------------

\bibitem{yoga-neti} Narayan Choyin Dorje. \textit{Manuale di yoga neti. Il tradizionale metodo yala neti per la pulizia delle vie nasali.} Tecniche nuove. 2008.


\bibitem{joythi-yoga} Swami Joythimayananda. \textit{Yoga}. Fratelli Frilli Editori. 2005. 


%----------------------------------------------------------------------------------------
%	DISTURBI DI VATA
%----------------------------------------------------------------------------------------

\bibitem{kumar} A . Vinaya Kumar. \textit{Ayurvedic Clinical Medicine}. Sri Satg\={u}ru
, Publications.

\bibitem{raghuram-manasa-def} Dr Raghuram Y.S. MD (Ay), Dr Manasa \textit{Vata disorders (Vatavyadhi): definitions, causes, symptoms}. Consultato in data Maggio 1, 2017 da https://easyayurveda.com/2017/06/05/vata-disorders-vatavyadhi/

\bibitem{raghuram-vata-causes} Dr Raghuram Y.S. MD (Ay) \textit{Different causes for Vata dosha imbalance, increase}. Consultato in data Maggio 1, 2017 da https://easyayurveda.com/2016/12/26/causes-vata-dosha-imbalance-increase/

\bibitem{raghuram-manasa-diff} Dr Raghuram Y.S. MD (Ay), Dr Manasa \textit{Difference between Vata Vyadhi and other pathological Vata manifestations}. Consultato in data Maggio 1, 2017 da https://easyayurveda.com/2017/06/05/vata-vyadhi-pathological-manifestation/


%----------------------------------------------------------------------------------------
%	LOMBALGIA - VISIONE MODERNA
%----------------------------------------------------------------------------------------

%statistiche mondiali
\bibitem{lombalgia-review} Hoy D, Bain C, Williams G, et al. \textit{A systematic review of the global prevalence of low back pain}. Arthritis Rheum. Vol. 64, nº 6, giugno 2012, pp. 2028–37, DOI:10.1002/art.34347, PMID 22231424.

\bibitem{anesthesiology} Vinod Malhotra; Yao, Fun-Sun F.; Fontes, Manuel da Costa, \textit{Yao and Artusio's Anesthesiology: Problem-Oriented Patient Management}, Hagerstwon, MD, Lippincott Williams \& Wilkins, 2011, pp. Chapter 49, ISBN 1-4511-0265-8.

\bibitem{globalburden} T Vos, \textit{Years lived with disability (YLDs) for 1160 sequelae of 289 diseases and injuries 1990–2010: a systematic analysis for the Global Burden of Disease Study 2010}.in Lancet, vol. 380, nº 9859, 15 dicembre 2012, pp. 2163–96, PMID 23245607.

\bibitem{gskreport} GSK Consumer Healthcare (GSK CH), \textit{RAPPORTO DI RICERCA. Risultati italiani sul dolore muscolo-scheletrico}. GLOBAL PAIN INDEX 2017. 


%----------------------------------------------------------------------------------------
%	LOMBALGIA - AYURVEDA
%----------------------------------------------------------------------------------------

%spiega la differenza tra trika shula e katigraha
\bibitem{insight-lom} Sajitha K. \textit{An INSIGHT in to “KATIGRAHA” (LOW BACK ACHE)}.Ancient Science of Life. 2001;21(1):16-17.


\bibitem{critical-review-lom} Gupta, S., Patil, V., \& Sharma, R. \textit{DIAGNOSIS AND MANAGEMENT OF KATISHOOLA (LOW BACK PAIN) IN AYURVEDA: A CRITICAL REVIEW}. AYUSHDHARA, 2017, 3(4).


%massaggio con sahacharadi taila per katishula
\bibitem{sahacha-massage-chronic-lom} Kumar S., Rampp T., Kessler C., Jeitler M., Dobos G. J., Lüdtke R., Meier L., and Michalsen A. \textit{Effectiveness of Ayurvedic Massage (Sahacharadi Taila) in Patients with Chronic Low Back Pain: A Randomized Controlled Trial}. The Journal of Alternative and Complementary Medicine. February 2017, 23(2): 109-115. doi:10.1089/acm.2015.0272.


%comparazione tra katibasti con mahanarayana ed esercizio fisico 
\bibitem{katibasti-esercizio-lom} Panda Ashok K., Debnath Saroj K. \textit{EFFECTIVENESS OF KATIVASTHI AND EXERCISE IN CHRONIC LOW BACK PAIN: A RANDOMIZED CONTROL STUDY}. International Journal of Research in Ayurveda \& Pharmacy, 2(2), 2011  338-342.

%efficacia di eranda muladi yapana basti
\bibitem{eranda-muladi-yapana-basti-lom} Damayanthie F. K. P., Thakar A.B., Shukla V.D. \textit{Clinical efficacy of Eranda Muladi Yapana Basti in the management of Kati Graha (Lumbar spondylosis)}. AYU 2013;34:36-41


%comparazione tra matra basti con/senza pinda sweda
\bibitem{sahacha-matra-basti-pindas-lom} Naveen B. \textit{A Comparative study of Sahachara Taila Matra Basti with and without Pinda Sweda in the management of Kati Shoola}. Dissertation, Department of Post Graduate Studies in Panchakarma S.J.G Ayurvedic Medical College \& Hospital, Koppal, 2012.


%comparazione katibasti, matra basti e patrapinda sweda
\bibitem{katibasti-matra-patrapindasweda-lom} Gupta, S., Sharma, R. \textit{Comparative clinical study of kati basti, patrapinda sveda and matra basti in kati shoola (low backache)}. Journal of Ayurveda and Holistic Medicine (JAHM), 2016, 3(6), 19-35.


%parisheka swedana con kheerabala taila
\bibitem{parisheka-swedana-ksheerabala-lom} Dilbag Singh J. \textit{ROLE OF PARISHEKA SVEDANA IN THE MANAGEMENT OF KATI GRAHA}. Dissertation, Rajiv Gandhi University of Health Science, Dept. of Post Graduate Studies in Panchakarma, S.D.M. College of Ayurveda and Hospital, Hassan, 2008.


%ruksha swedana, risultati non ottimi ma rilevanti
\bibitem{kolakulattadi-ruksha-sveda-lom} Anjaly N.V. \textit{A STUDY ON THE EFFECT OF SWEDANA WITH KOLAKULATHADI CHOORNA AND VALUKA IN KATIGRAHA}. Dissertation, Rajiv Gandhi University of Health Science, Dept. of Post Graduate Studies in Panchakarma K.V.G. Ayurvedic Medical College and Hospital, Sullia, 2010.

%----------------------------------------------------------------------------------------
%	SCIATICA
%----------------------------------------------------------------------------------------

\bibitem{agnikarma} Bakhashi B, Gupta SK, Rajagopala M, Bhuyan C. \textit{A comparative study of Agni karma with Lauha, Tamra and Panchadhatu Shalakas in Gridhrasi (Sciatica)} AYU. 2010;31:240–4

\bibitem{kabasti} Padmakiran C., Ramachandra R., Prasad U. N., Rao Niranjan1. \textit{Role of Katibasti in Gridhrasi (Lumbago Sciatica Syndrome)} International Ayurvedic Medical Journal, ISSN:2320 5091.


\bibitem{agni-kabasti} Bali Y, Vijayasarathi R, Ebnezar J, Venkatesh B. \textit{Efficacy of Agnikarma over the padakanistakam (little toe) and Katibasti in Gridhrasi: A comparative study}. Int J Ayurveda Res. 2010;1:223–30. 

\bibitem{agni-siraveda} Vaneet Kumar J, Dudhamal TS, Gupta SK, Mahanta V. \textit{A comparative clinical study of Siravedha and Agnikarma in management of Gridhrasi (sciatica)} . Ayu. 2014 Jul-Sep;35(3):270-6. doi: 10.4103/0974-8520.153743. 

\bibitem{rakta-kabasti} Singh SK, Rajoria K, Sharma RS. \textit{Comparative efficacy of Raktamokshana and Kativasti in Gridhrasi (w.s.r. to sciatica)} J Ayurveda. 2010;4:51–67.

\bibitem{reviews-grid} Arora V, Dudhamal TS, Gupta SK, Mahanta VD.\textit{ Review of researches on Grudhrasi (Sciatica)}. At IPGT and RA, Jamnagar. Indian J Ancient Med Yoga. 2013;6:31–6.

\bibitem{type-grid} Parashar S., Pandya D.H. \textit{A critical review of Vataja and Vatakaphaja Gridhrasi} International Ayurvedic Medical Journal ISSN:2320 5091

\bibitem{critical-analysis-grid} Subina S., Ananthram S., Prathibha C.K. \textit{Critical Anaalysis of Gridhrasi Chikitsa} Ayurpharm Int J Ayur Alli Sci., Vol. 5, No. 9 (2016) Pages 121 – 127, ISSN: 2278-4772.

\bibitem{tesi1} Rita V. Khagram. \textit{A comparative study of Kati Basti
and Matra Basti in the management of G\d{r}idhra\={s}i (Sciatica)}. Gujarat Ayurved University. 2004.
 

\bibitem{kabasti-eranda-grid} Greeshma G. \textit{Comparative study of Katibasti with Erandamoola Kashaya and Eranda Taila in Gridhrasi}. Dissertation, Rajiv Gandhi University of Health Science, Dept. of Panchakarma S.D.M.C.A, Karnataka, Bangalore. 2010-2011.


%qiu spiega i trattamenti - tabelle trattamenti x i due tipi di sciatica - terapie snehana e swedana
\bibitem{pancaka-grid} Imlikumba, BA Lohith, Parappagoudra Mahesh, K Singha, S Lalravi. \textit{Role of
Panchakarma in management of Gridhrasi}. J Ayurveda Integr Med Sci 2016;2:92-97.


\bibitem{management-grid} Satya P. and Sarvesh K. S. \textit{Ayurvedic Management for Gridhrasi with Special Reference to Sciatica- A Case Report}. International Journal of Advanced Ayurveda, Yoga, Unani, Siddha and Homeopathy 2015, Volume 4, Issue 1, pp. 262-265, Article ID Med-240, ISSN: 2320 – 0251.




%----------------------------------------------------------------------------------------
%	RICETTE
%----------------------------------------------------------------------------------------



%----------------------------------------------------------------------------------------
%	TRATTAMENTI
%----------------------------------------------------------------------------------------

\bibitem{tiwari} Maya Tiwari (Sri Swamini Mayatitananda). \textit{I Segreti della Guarigione Ayurvedica. La guida più completa all'antica medicina indiana}. Edizioni il Punto d'Incontro. 1998-2006.



\bibitem{joythi-abhyanga} Swami Joythimayananda. \textit{Abyangam - Massaggio Ayurvedico}. Fratelli Frilli Editori. 2007. 




\end{thebibliography} 





\end{document}
