%!TEX root = ../tesi-ayurveda-v1.tex

\chapter*{Glossario} 


\begin{longtable}{|p{3cm}|p{3cm}|p{10cm}|}
	\hline
    {\bf Sanscrito} & {\bf Sanscrito (diacritico)} & {\bf Italiano} \\ \hline

	\endhead % all the lines above this will be repeated on every page

%\rowcolor{celeste}

Acharya & \={A}ch\={a}rya & Insegnante \\
Adhikara & Adhikara & Indicazione primaria \\
Agnimandya & Agnim\={a}ndya & Lentezza della digestione \\
Alpata & Alpat\={a} & Insignificante - minuscola \\
Anupana & Anup\={a}na & Veicolo fluido in una medicina \\
Anuvasana & Anuv\={a}sana & Enema oleoso \\
Apana & Ap\={a}na & Uno dei cinque tipi di vata con direzione verso il basso \\
Arocaka & Arocaka & Indigestione - desiderio o mancanza di appetito \\
Bhaktadvesa & Bhaktadve\d{s}a & Perdita di appetito \\
Dharana & Dh\={a}ra\d{n}\={a} & Concentrazione \\
Dosha & Do\d{s}a & Umore \\
Dushti & Du\d{s}\d{t}i & Corruzione \\
Gaurava & Gaurava & Pesantezza \\
Grdhra & G\d{r}dhra & Avvoltoio \\
Gridhra\={s}i & G\d{r}idhra\={s}i & Sciatica \\
Gridhrasi & G\d{r}idhra\={s}i & Sciatica \\
Hani & H\={a}ni & Perdita - diminuizione \\
Jalauka & Jalauk\={a} & Sanguisuga \\
Jangha & Ja\.{n}gh\={a} & Coscia \\
Janghapindi & Ja\.{n}gh\={a}pi\d{n}\d{d}\={\i} & Polpaccio \\
Janu & J\={a}nu & Ginocchio \\
Kandara & Ka\d{n}\d{d}ar\={a} & Tendine - legamento \\
Kati & Ka\d{t}\={\i} & Parte bassa della schiena - vita \\
Kriya & Kriy\={a} & Azione del corpo \\
Ksheera & K\d{s}\={\i}ra & Latte \\
Laksana & Lak\d{s}a\d{n}a & Sintomo \\
Lashuna & La\'{s}una & Aglio \\
Madhuyasti & Madhuya\d{s}\d{t}i & Liquirizia \\
Mukha-praseka & Mukha-praseka & Salivazione \\
Nidana & Nid\={a}na & Cause primarie \\
Niruha & Nir\={u}ha & Enema non di tipo oleoso \\
Pada & P\={a}da & Piede \\
Paka & P\={a}ka & Cuocere \\
Pancha & Pa\~{n}ca & Cinque \\
Prayoga & Prayoga & Uso terapeutico \\
Pristha & P\d{r}\d{s}\d{t}ha & didietro - back of the body \\
Pristhabhaga & P\d{r}\d{s}\d{t}abh\={a}ga & Parte posteriore \\
Samhita & Sa\d{m}hit\={a} & Raccolta ordinata di testi e versi \\
Sevana & Sevana & Pratica \\
Shakti & \'{S}akti & Energia - forza - abilità \\
Shula & \'{S}\={u}la & Dolore \\
Siravedha & Sir\={a}vedha & Veno-puntura \\
Snehapana & Snehap\={a}na & Oleazione interna \\
Spandana & Spandana & dolore pulsante \\
Sphic & Sphic & Didietro - posteriore \\
Stambha & Stambha & Rigidità - durezza - inflessibilità \\
Tandra & Tandr\={a} & Fiacchezza - mancanza di energie \\
Toda & Toda & Dolore acuto - dolore penetrante \\
Uru & \={U}ru & Gamba \\
Vacarana & Vacarana & Prelievo \\
Vata & V\={a}ta & Vento \\
Visravana & Visr\={a}va\d{n}a & Che causa perdita di sangue \\
Viswaci & Vi\'{s}w\={a}ci & Disturbo simile alla sciatica ma che interessa gli arti superiori \\
Vyadhi & Vy\={a}dhi & Malattia - disordine \\
Vyana & Vy\={a}na & Una delle cinque arie vitali \\
Vyayama & Vy\={a}y\={a}ma & Esercizio fisico \\ \hline
   
\end{longtable}


%----------------------------------------------------------------------------------------
%	XTAB - COMMENTATO
%----------------------------------------------------------------------------------------


\begin{comment}


\begin{center}

\tablefirsthead{
    \hline
    \multicolumn{1}{|l@{\extracolsep{\fill}}|}{\textbf{Sanscrito}} &
    \multicolumn{1}{l@{\extracolsep{\fill}}|}{\textbf{Sanscrito (diacritico)}} &
    \multicolumn{1}{l|}{\textbf{Italiano}} \\

    \hline
 }

\tablehead{
	%\multicolumn{3}{c}{\textbf{Tabella \thetable{} -- continua dalla pagina precedente} }\\
	\hline
	\multicolumn{1}{|l@{\extracolsep{\fill}}|}{\textbf{Sanscrito}} &
    \multicolumn{1}{l@{\extracolsep{\fill}}|}{\textbf{Sanscrito (diacritico)}} &
    \multicolumn{1}{l|}{\textbf{Italiano}} \\

	\hline }


\tablelasthead{
	%\multicolumn{3}{c}{\textbf{Tabella \thetable{} -- termina dalla pagina precedente} }\\
  	\hline 
  	\multicolumn{1}{|l@{\extracolsep{\fill}}|}{\textbf{Sanscrito}} &
    \multicolumn{1}{l@{\extracolsep{\fill}}|}{\textbf{Sanscrito (diacritico)}} &
    \multicolumn{1}{l|}{\textbf{Italiano}} \\

   	\hline }


\tabletail{%
	\hline
	\multicolumn{3}{|r|}{\small\sl continua alla pagina successiva}\\
	\hline}


\tablelasttail{\\ \hline}


\begin{xtabular*}{\textwidth}{|l@{\extracolsep{\fill}}|l@{\extracolsep{\fill}}|l|}


Acharya & \={A}ch\={a}rya & Insegnante \\
Adhikara & Adhikara & Indicazione primaria \\
Agnimandya & Agnim\={a}ndya & Lentezza della digestione \\
Alpata & Alpat\={a} & Insignificante - minuscola \\
Anupana & Anup\={a}na & Veicolo fluido in una medicina \\
Anuvasana & Anuv\={a}sana & Enema oleoso \\
Apana & Ap\={a}na & Uno dei cinque tipi di vata con direzione verso il basso \\
Arocaka & Arocaka & Indigestione - desiderio o mancanza di appetito \\
Bhaktadvesa & Bhaktadve\d{s}a & Perdita di appetito \\
Dharana & Dh\={a}ra\d{n}\={a} & Concentrazione \\
Dosha & Do\d{s}a & Umore \\
Dushti & Du\d{s}\d{t}i & Corruzione \\
Gaurava & Gaurava & Pesantezza \\
Grdhra & G\d{r}dhra & Avvoltoio \\
Gridhra\={s}i & G\d{r}idhra\={s}i & Sciatica \\
Gridhrasi & G\d{r}idhra\={s}i & Sciatica \\
Hani & H\={a}ni & Perdita - diminuizione \\
Jalauka & Jalauk\={a} & Sanguisuga \\
Jangha & Ja\.{n}gh\={a} & Coscia \\
Janghapindi & Ja\.{n}gh\={a}pi\d{n}\d{d}\={\i} & Polpaccio \\
Janu & J\={a}nu & Ginocchio \\
Kandara & Ka\d{n}\d{d}ar\={a} & Tendine - legamento \\
Kati & Ka\d{t}\={\i} & Parte bassa della schiena - vita \\
Kriya & Kriy\={a} & Azione del corpo \\
Ksheera & K\d{s}\={\i}ra & Latte \\
Laksana & Lak\d{s}a\d{n}a & Sintomo \\
Lashuna & La\'{s}una & Aglio \\
Madhuyasti & Madhuya\d{s}\d{t}i & Liquirizia \\
Mukha-praseka & Mukha-praseka & Salivazione \\
Nidana & Nid\={a}na & Cause primarie \\
Niruha & Nir\={u}ha & Enema non di tipo oleoso \\
Pada & P\={a}da & Piede \\
Paka & P\={a}ka & Cuocere \\
Pancha & Pa\~{n}ca & Cinque \\
Prayoga & Prayoga & Uso terapeutico \\
Pristha & P\d{r}\d{s}\d{t}ha & didietro - back of the body \\
Pristhabhaga & P\d{r}\d{s}\d{t}abh\={a}ga & Parte posteriore \\
Samhita & Sa\d{m}hit\={a} & Raccolta ordinata di testi e versi \\
Sevana & Sevana & Pratica \\
Shakti & \'{S}akti & Energia - forza - abilità \\
Shula & \'{S}\={u}la & Dolore \\
Siravedha & Sir\={a}vedha & Veno-puntura \\
Snehapana & Snehap\={a}na & Oleazione interna \\
Spandana & Spandana & dolore pulsante \\
Sphic & Sphic & Didietro - posteriore \\
Stambha & Stambha & Rigidità - durezza - inflessibilità \\
Tandra & Tandr\={a} & Fiacchezza - mancanza di energie \\
Toda & Toda & Dolore acuto - dolore penetrante \\
Uru & \={U}ru & Gamba \\
Vacarana & Vacarana & Prelievo \\
Vata & V\={a}ta & Vento \\
Visravana & Visr\={a}va\d{n}a & Che causa perdita di sangue \\
Viswaci & Vi\'{s}w\={a}ci & Disturbo simile alla sciatica ma che interessa gli arti superiori \\
Vyadhi & Vy\={a}dhi & Malattia - disordine \\
Vyana & Vy\={a}na & Una delle cinque arie vitali \\
Vyayama & Vy\={a}y\={a}ma & Esercizio fisico


\end{xtabular*}
\end{center}


\tablefirsthead{%
\hline
\multicolumn{1}{|c}{\tbsp Number} &
\multicolumn{1}{c}{Number$^2$} &
Number$^4$ &
\multicolumn{1}{c|}{Number!} \\
\hline}

\end{comment}


